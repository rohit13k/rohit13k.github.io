\documentclass[a4paper]{article}

\usepackage[a4paper,margin=2cm]{geometry}
\usepackage{graphics}
\usepackage{graphicx}

\thispagestyle{empty}


\begin{document}
\section*{Teaching Statement---Toon Calders}

\subsection*{A strong basis to build upon}
Few companies have activities which are solely concentrated around software development. Therefore, if a university wants to form engineers that will later on take leading roles in enterprises, it will have to equip its students with the necessary tools to understand and reason about the main activities of the company. They should be able to develop a vision that goes beyond the purely technical computer science aspects. Therefore, it is my strong believe that a good engineer has a broad background, not only in technical computer-related aspects, but also in other related disciplines such as physics, mathematics, electronics, and biology. Next to these technical disciplines also soft skills such as the ability to clearly communicate both written and orally, and working in group are important. These competencies, however, cannot be acquired through a single course but need to be deeply embedded into the different courses.

\subsection*{Specialization based on solid research}
Studies in the Master years should be more specialized and be closely connected to research and be supported by specialists in the field. Even though engineering studies are often heavily tilted towards applying and developing existing techniques, theoretical background is needed to build a deep understanding. For example, when lecturing about noSQL databases it is in my opinion more important to talk about the CAP theorem\footnote{The CAP theorem states that it is impossible for a distributed storage system to guarantee at the same time more than two of the following properties: data consistency, availability of the database and partition tolerance.} and its implications for distributed database systems than covering all noSQL database systems. Today's hype system may be gone tomorrow, but the CAP theorem will still hold. Furthermore, even though it is important that the curriculum is up to date, investing time on lecturing fundamental concepts is never a waste of time. An excellent example is that of Deductive databases; this database paradigm was considered outdated, but a lot of the techniques and concepts are fundamentally different from the traditional relational viewpoint. The recent re-emergence of deductive database concepts in the context of the semantic web confirms that strong concepts are never dead. To realize a good balance between fundamental concepts and an up-to-date curriculum, lecturers need to be specialists in their domain in order to have a broad overview and being able to distinguish fundamentally new concepts from hype.

\subsection*{Reaching excellence through internationalization and collaboration}
Closely connected to the previous two points is internationalization and collaboration. International experiences such as Erasmus exchanges are on the one hand great opportunities for students to specialize in a field where the specialists are not locally available, and on the other hand build maturity and independence. The exposure to different traditions and cultures in teaching is an enrichment both for the student and the receiving institute. Furthermore, within the IT4BI Erasmus Mundus Master course I am closely involved in, I have experienced first-handed the benefits of international collaboration in education. Offering the specialization in Business Intelligence over different institutes tremendously increases the pool of specialists and hence the number of high-quality specialized courses that can be offered and at the same time it enlarges the scale by attracting more students. Furthermore, the different perspectives of the institutes and the necessity to coordinate with respect to course content and focus of the Master have led to an increased awareness of quality.

\subsection*{Prospective Courses}
Currently, at ULB, I am teaching two 5 ECTS courses on Business Process Management and Data Warehousing. In the past, at the TU/e I have been teaching Advanced Databases and Data Mining. Furthermore, as a guest lecturer I recently lectured a short course on data stream processing. I think most of these courses would fit very well in a Data Science Master at the TU/e. A data science master should touch upon three aspects: data management, data analysis, and business intelligence. For many of these aspects a lot of expertise is already present at TU/e; a strong statistics group and a visual analytics group (data analysis), business process management and process mining (business intelligence), databases and information retrieval (data management). I envision contributing courses, mainly belonging to the intersection data management and data analysis, to complete this picture. Depending on the size of the data mining team, for instance the following courses could be offered: Data Warehousing (how to store, maintain, and query large amounts of data; the more traditional techniques; OLAP), Big data management (more recent developments in the field; distributed systems such as Hadoop), Data stream management (efficient techniques for managing data streams other than scaling out on distributed systems), Data mining and machine learning, a Business Intelligence Seminar.
\end{document} 