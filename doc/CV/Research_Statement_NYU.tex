	\documentstyle[11pt]{article}

\begin{document}

\title{Research Statement}
\author{Rohit Kumar}
\maketitle

\section{Introduction}

\section{An Example}
As an example, Thomas Cover has proven that his Universal Portfolio investment
strategy ``learns'' to perform almost as well as the best of a class of
investment strategies called constant rebalanced portfolios (CRPs) \cite{Cover}.  A CRP
keeps the same distribution of wealth among a group of stocks every day.  For
example, one CRP rebalances each day so that it has $1/4$ of its value in
stock A and $3/4$ in stock B.  If there are $n$ stocks, the set of CRPs is an
$n$-dimensional simplex where the $i$th component represents the proportion of
value in the $i$th stock.  Typically, in hindsight, some of the CRPs perform
very well, and therefore the Universal Portfolio also performs well.
Universal Portfolios are especially interesting because they can be used for
data compression and language models, where the guarantees are more striking.

The Universal Portfolio each day rebalances according to an average (integral)
over the set of CRPs, where the weights are proportional to the performance of
each constituent CRP.  However, evaluating this integral in a straightforward
way or by uniform sampling requires time $O(t^{n-1})$ where $t$ is the number
of days and $n$ is the number of stocks.  Since this is exponential in the
number of stocks, previous experiments have been limited to two stocks
\cite{Cover, BK, Helmbold}.

Using an idea of Frieze and Kannan \cite{FK}, we have shown how one can
efficiently sample CRPs to compute this integral.  The idea is to start at an
arbitrary CRP and take a random walk that is biased towards high-performing
CRPs.  The stationary distribution of this walk is exactly the distribution of
the Universal Portfolio, and the underlying Markov chain is rapidly mixing.
As a result, we can approximate the components of the Universal Portfolio with
a total amount of computation polynomial in both $t$, the number of days, and
$n$, the number of stocks \cite{KV}.


\section{Target Problems}
Since many learning problems can be solved by such weighted averages of
learners, there are many problems for which a random walk may give a
computationally efficient solution.  Unfortunately, for each problem, one
needs to define a new random walk and prove that it is rapidly mixing, which
is often difficult.  One interesting direction is to try and provide tools for
showing that classes of learning problems have walks that are rapidly mixing. 

However, we also have in mind several specific problems for which we would
like efficient algorithms.  One such problem is that of learning decision
lists.  Decision lists are a simple and natural class of boolean functions.
The input is a set of items which each have say $n$ boolean properties.  This
may be most easily demonstrated with an example decision list:
\begin{verbatim}
If (Property 6=True) then True,
else If (Property 4=True) then False,
else If (Property 8=False) then False,
else False.
\end{verbatim}
		
Decision lists may learned from a relatively small amount of data.  If the
list has length $l$ and there are $n$ properties, a straightforward algorithm
efficiently learns with $O(ln)$ examples of data.  However, there is an
algorithm that learns from $O(l \log n)$ examples of data, but is not
computationally efficient.  Since the goal of learning is to predict, we must
describe how to predict the value of the function on new data, from previous
examples of input-output pairs of the function.  In several models of
learning, it can be shown that if you predict the value of the function
according to a majority vote over all decision lists consistent with the past
data, then you will learn from $O(l \log n)$ examples of data. 

Unfortunately, there are $(2n)^l$ decision lists of length $l$, so there may
be too many consistent decision lists to enumerate.  Instead, we would like
to take a random walk on the set of consistent decision lists to efficiently
estimate the average prediction on a particular input.  In the case where the
decision lists are almost evenly split on their prediction, it is acceptable
if we do not agree with the majority vote.

There are also potential application of random walks to adaptive data
structures.  These data structures adapt to a sequence of accesses and
therefore, in a sense, learn.  It would be remarkable to have optimal data
structure that updates itself by taking random walks, and we believe this may
be possible for binary search trees, for which there are no known dynamically
optimal data structures.


\begin{thebibliography}{99}

\bibitem{BK}
\newblock A.~Blum and A.~Kalai.
\newblock Universal Portfolios with and without Transaction Costs.
\newblock {\em Machine Learning}, 35:3, 1999.

\bibitem{Cover}
\newblock T.~Cover.
\newblock Universal portfolios.
\newblock {\em Math. Finance}, 1(1):1-29, January 1991.

\bibitem{FK}
\newblock A. Frieze and R. Kannan.
\newblock Log-Sobolev inequalities and sampling from log-concave distributions.
\newblock Annals of Applied Probability 9, 14-26.

\bibitem{Helmbold}
\newblock D.~Helmbold, R.~Schapire, Y.~Singer, and M.~Warmuth.
\newblock On-line portfolio selection using multiplicative updates. 
\newblock {\em Machine Learning: Proceedings of the Thirteenth 
International Conference}, 1996.

\bibitem{KV}
\newblock A.~Kalai and S.~Vempala.
\newblock Efficient Algorithms for Universal Portfolios.
\newblock To appear in {\em Proceedings of the 41st Annual Symposium on the
Foundations of Computer Science (FOCS '00)}, Redondo Beach, 2000. 

\bibitem{KLS}
\newblock R. Kannan, L. Lovasz and M. Simonovits.
\newblock Random walks and an $O^*(n^5)$ volume algorithm for convex bodies.
\newblock Random Structures and Algorithms 11, 1-50.

\bibitem{LW}
\newblock N.~Littlestone and M.~Warmuth.  
\newblock The weighted majority algorithm.
\newblock {\em Information and Computation} 108(2):212-261, 1994.

\bibitem{Schapire}
\newblock R.~Schapire.  
\newblock The strength of weak learnability.  
\newblock {\em Machine Learning},
5(2):197-227, 1990. 

\bibitem{VC}
\newblock V.~Vapnick and A.~Chervonekis.  On the uniform convergence of
relative frequencies of events to their probabilities.  {\em Theory of
Probability and its Applications}, 16(2):264-280, 1971.


\end{thebibliography}

\end{document}


