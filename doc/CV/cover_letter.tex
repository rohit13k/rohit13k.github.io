
\section*{Cover Letter}
A wise man once said: “Data is a precious thing and will last longer than the systems themselves”. That man was Tim Berners-Lee.

Indeed, we are living in the world where Data is all that matters: how to acquire, store, transfer, transform and generate informative knowledge out of it. For this reason, and for many others, my career objective is to act as an expert in data management.

My current fascination with data management and data analysis comes from my diverse background. In my formative years, I was very curious on understanding how everything works which lead me to pursue graduation in Physics from the prestigious Hindu College in Delhi University. While I was doing my graduation, my curiosity to learn new things took me to do a 3-year summer internship in one of the most prestigious research institutes of India (Jawaharlal Nehru Centre for Advanced Scientific Research, JNCASR IISC Bangalore). I used to visit the research center every summer vacation for 1.5 months to do an internship as a research assistant to study Nanomaterials under the supervision of Prof CNR Rao as part for the POCE (Project-oriented chemical education) internship.

After finishing my graduation, I joined TCS (Tata consultancy services) and was working as a research and training assistance while doing my masters from CMI (Chennai Mathematical Institute). It was a joint program conducted by TCS and CMI wherein I use to go to university for my Master in Computer science course 2 days a week and work as research and training assistant in TCS for rest of the time.
Always eager to try something new and challenging lead me to a new journey where I took upon myself to solve one of the complex problems faced by our organization. The problem was to create a highly scalable, easy to use and setup, ultra-secure and reliable platform for conducting All India level online browser-based examinations. There were many challenges to conducting an online exam in India, infrastructure problems, network issues etc. I took it as my Master thesis project and with the help of some of my colleagues created a platform which was used to conduct exam for 80,000 students at the same time at 854 different locations all over India. We patented the overall platform and all the innovations which went into the platform. After finishing the platform and my master’s I moved to TCS- Mumbai where I worked for 3 more years on taking this platform from internal software to a cloud-based service platform for Indian education and banking market. Currently, the platform is being used for all the high level major Online Exams in India. During this 5 year project I realized the importance of scalability and got my first taste on high scale large systems.

While working, it was impossible to ignore the fact that the world is advancing at a tremendously accelerated pace and paradigm shifts will arrive at increasingly shorter intervals, specifically in the field of Data Management, where changes are very apparent. Companies are creating large databases and streaming huge amounts of data, but missing the expertise of analysis and advisory management. To
cope with all the changes, my first step was to apply for IT4BI (Information Technology for Business Intelligence) Ph.D. program, where I was granted an Erasmus Mundus scholarship.

For the last 3.5 years, I have been doing research at top-notch universities across Europe, enriching my skills in Big Data, Stream Mining and Distributed processing. My research focuses on designing efficient one-pass algorithms to analysis large interaction networks such as social media network. I have also worked on Influence maximization and Information propagation problems using Interaction network. Currently, I am working on distributed graph processing, where I focus on using cost model-based approach for adaptive graph partitioning on Apache GraphX. My current experience working with the Big data technologies and my past experience with working real-world large-scale systems motivated me to look for opportunities which could utilize both my research and industry experience. In this regard, Telefonica research labs seemed a perfect fit as apart from doing valuable research and publications it also presents the opportunity to apply the research to real-world problems.

To conclude, I believe I have got what it takes to make a valuable contribution to my team as I always do, and reflect my solid grounding for the benefit of my ultimate goal, that is doing what it takes to improve myself and my surrounding, and given the chance to work with Telefonica will put me on the right track.